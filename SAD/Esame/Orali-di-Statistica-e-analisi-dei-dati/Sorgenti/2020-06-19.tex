\documentclass{article}
\usepackage[T1]{fontenc}
\usepackage[italian]{babel}
\usepackage[margin=2cm]{geometry}
\usepackage[utf8]{inputenc}
\usepackage{amsmath}

\newcommand{\bb}{\textbf}
\newcommand{\ii}{\textit}

\title{Orali del 19 giugno 2020}
\author{Alessandro Di Gioacchino}

\begin{document}

    \maketitle

    A cosa serve un classificatore binario? Come faccio a valutarne la performance? Quale potrebbe essere una matrice di confusione ideale? Approssimativamente una diagonale, con
    tutti i valori che non sono sulla diagonale molto vicini a zero. Cosa usiamo se volessimo 'comprimere' la matrice in un numero più basso di valori? Cosa ci racconta la
    specificità? Esprime sempre le volte in cui il classificatore ha indovinato? La specificità, che ha veri positivi al numeratore e veri positivi più falsi negativi al
    denominatore, è la probabilità di quale evento? È la probabilità che, se il classificatore risponde “è positivo”, l'oggetto in esame è effettivamente positivo. In altri 
    termini, ci dice quale frazione di casi positivi il classificatore riesce ad indovinare? Di cosa tratta l'altro indice, la sensitività? Idealmente, sarei contento per quali
    valori di sensitività e specificità? Curva ROC. Perché il punto $ ( 0 , 1 ) $ indica il classificatore ideale? Cosa posso dire del numero di falsi positivi e del numero di
    falsi negativi se il classificatore si posiziona in tale punto? A cosa serve la diagonale nella grafico della curva? È possibile ottenere una curva degenere che giace su di
    essa? Sì, con un classificatore che sceglie in modo casuale. \\
    Conosciamo solo il grafico della funzione di ripartizione di due v.a. \\
    Disegna il grafico della funzione di ripartizione di una v.a. discreta. Cosa si intende con il simbolo di intervallo aperto all'inizio di ciascun tratto? Tracciamo nello stesso
    grafico la funzione di ripartizione di una v.a. continua, definita sull'intervallo da zero a tre (come le specificazioni della v.a. discreta di cui sopra). Anzi, riferiamoci ad
    una v.a. uniforme con specificazioni zero, uno, due. Guardando questo grafico, possiamo dire qualcosa sulla relazione tra i valori attesi delle due variabili aleatorie? Quella
    discreta potrebbe essere una binomiale? Quali specificazione ha? Una variabile con specificazioni uno, due, tre può essere una binomiale? No, dovrebbe esserci anche zero. Il
    ragionamento che dobbiamo fare coinvolge solo quanto il grafico esprime visivamente. Possiamo legare il grafico della funzione di ripartizione al valore atteso? Detto questo,
    siamo in grado di dire se c'è una particolare relazione tra il valore atteso della v.a. continua e della v.a. discreta? Sì: il valore atteso della v.a. discreta è maggiore. \\
    Abbiamo un campione aleatorio descritto da $ n $ v.a., $ X_1 , \dots , X_n $, e vogliamo stimare il valore atteso della popolazione. Normalmente cosa useremmo? Proviamo invece
    a stimarlo con la varianza campionaria. Qual è la sua definizione? Supponendo di avere già il valore della varianza campionaria, in quali situazioni usarlo per stimare il
    valore atteso potrebbe avere senso? Ce ne potrebbero essere alcune in cui tale stimatore non è neanche deviato, per esempio se la distribuzione ha valore atteso e varianza
    uguali. Quale distribuzione ha questa caratteristica? Se il valore atteso è invece funzione della varianza, lo stimatore potrebbe essere deviato: perché? Pensiamo
    all'esponenziale, in cui la varianza è il quadrato del valore atteso: in questo caso, per ottenere il valore atteso data la varianza campionaria, ne dovremmo estrarre la
    radice quadrata. Cosa possiamo dire circa le proprietà di questo stimatore? Cosa vuol dire che uno stimatore è corretto? Qual è il valore atteso di $ S $, cioè della deviazione
    standard campionaria (radice della varianza campionaria)? $ 1 / \lambda $ \\
    Sappiamo quindi che $ \mathcal E [ S ] = \frac{ 1 }{ \lambda } $, perché $ S^2 $ è uno stimatore non deviato della varianza. Come indichiamo analiticamente questa cosa?
    Sappiamo che la varianza campionaria è sempre uno stimatore non deviato della varianza della popolazione. A partire dalla seconda equazione,
    $ \mathcal E [ S^2 ] = \frac{ 1 }{ \lambda^2 } $, possiamo ricavare la prima? No, perché il valore atteso è lineare. Tornando al punto di partenza, possiamo quindi dire che la
    varianza campionaria è uno stimatore non deviato del valore atteso in quali casi? Quando la trasformazione che porta la varianza nel valore atteso è lineare, perché basta
    applicare la stessa trasformazione allo stimatore della varianza. \\

\end{document}