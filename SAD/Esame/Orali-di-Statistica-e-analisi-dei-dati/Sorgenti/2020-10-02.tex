\documentclass{article}
\usepackage[T1]{fontenc}
\usepackage[italian]{babel}
\usepackage[margin=2cm]{geometry}
\usepackage[utf8]{inputenc}
\usepackage{amsmath}

\newcommand{\bb}{\textbf}
\newcommand{\ii}{\textit}

\title{Orali del 2 ottobre 2020}
\author{Alessandro Di Gioacchino}

\begin{document}

    \maketitle
    
    Teorema di Bayes. \\
    $ P ( A \mid B ) $ vuole quantificare il rapporto di causalità tra $ A $ e $ B $? No. Il teorema di Bayes stesso indica che non è stato $ B $ a scatenare $ A $, perché considera
    anche $ P ( B \mid A ) $ \\
    \[
        P ( A \mid B ) = \frac{ P ( B \mid A ) P ( A ) }{ P ( B ) }
    \]
    Esercizio: ho un'urna con $ 10 $ palline, di cui $ 5 $ bianche e le altre nere. Estraiamo (senza re-immissione) due palline: $ E_1 $ è l'esito della prima estrazione, $ E_2 $
    l'esito della seconda. Qual è la probabilità che $ E_2 $ sia bianca, dato che $ E_1 $ è bianca? \\
    Stimatore non deviato per il parametro di un modello bernoulliano. \\
    
    Abbiamo un campione in cui ogni osservazione è una coppia. Vogliamo calcolare le frequenze congiunte di questo campione. Dopo averle calcolate, come le organizzeremmo per
    mostrarle? La matrice in cui organizziamo le frequenze è di dimensioni $ n \cdot n $, posto che $ n $ sia la taglia del campione? No, è una matrice $ a \cdot b $ dove $ a $ sono i
    valori osservabili del primo carattere e $ b $ quelli del secondo. Cosa indica un generico elemento della matrice? Se indicasse la frequenza totale in cui queste coppie si ripetono,
    di che frequenza staremmo parlando? Assoluta. Che succede se sommo una colonna di quella matrice? Ottengo la frequenza marginale di un carattere della coppia; in altri 
    termini, stiamo tenendo costante un elemento della coppia e facendo variare l'altro. \\
    Immaginiamo di avere una coppia di variabili aleatorie. Il concetto di valore atteso si può estendere in modo da coinvolgerle entrambe: come? \\
    Il valore atteso di una variabile aleatoria è aleatorio o costante? È costante. Un'espressione che coinvolge qualcosa di aleatorio ha come risultato qualcosa di aleatorio:
    quindi, siccome il valore atteso di una variabile aleatoria è costante, non compare alcuna variabile aleatoria nel calcolo di $ \mathcal E ( X + Y ) $ \\
    Come calcoliamo $ \mathcal E ( X + Y ) $ \\
    In quale caso possiamo scrivere la funzione di massa congiunta come prodotto delle marginali?
    Nel seguente passaggio, scriviamo la doppia sommatoria di una somma come la somma di due sommatorie:
    \[
        \sum_{ x } \sum_{ y } ( x P ( X = x ) + y P ( Y = y )) = \sum_{ x } \sum_{ y } x P ( X = x ) + \sum_{ x } \sum_{ y } y P ( Y = y )
    \]
    Nel caso più generale possibile, cosa cambia se al posto della somma avessimo una generica funzione $ g $? \\
    Vogliamo stimare il valore atteso di una popolazione. Cos'è $ \tau (\Theta) $? Una quantità ignota che dipende dal parametro $ \Theta $ \\
    Il valore atteso che vogliamo stimare è $ \Theta $ o $ \tau (\Theta) $? Sicuramente è $ \tau (\Theta) $, eventualmente anche $ \Theta $ (se il parametro della distribuzione è
    il valore atteso). Come stimiamo il valore atteso? Come ottengo la stima? Calcolando lo stimatore. Che stimatore utilizziamo di solito per il valore atteso? Com'è definita la
    media campionaria? Perché utilizziamo proprio lei? Dimostriamo che è uno stimatore non deviato per il valore atteso di una popolazione. Perché è desiderabile che uno stimatore
    sia non distorto? Cosa vuol dire che il valore atteso della popolazione è esattamente quanto voglio stimare? Cosa posso dire di una distribuzione, se conosco il suo valore
    atteso?
    
\end{document}